\documentclass[12pt, letterpaper, final, onecolumn, titlepage] {article}

\usepackage{enumerate}
\usepackage{graphicx}
\usepackage{listings}
\usepackage{color}
\usepackage{setspace}
\usepackage[margin=1in]{geometry}
\usepackage{mathtools}
\usepackage{amsmath}
\usepackage{hyperref}

\title{ECE 528: Cloud Computing \\
	\vspace{1.5cm}
   		\begin{center}\includegraphics{umlogo} \end{center}
	\vspace{1.5cm}
	\textbf{Assignment \#3} \\
Python Image Download and Display}
	
\author{Michael Bowyer}

\date{\today}

\definecolor{dkgreen}{rgb}{0,0.6,0}
\definecolor{gray}{rgb}{0.5,0.5,0.5}
\definecolor{mauve}{rgb}{0.58,0,0.82}

\lstset{frame=tb,
  language=C,
  aboveskip=3mm,
  belowskip=3mm,
  showstringspaces=false,
  columns=flexible,
  basicstyle={\small\ttfamily},
  numbers=none,
  numberstyle=\tiny\color{gray},
  keywordstyle=\color{blue},
  commentstyle=\color{dkgreen},
  stringstyle=\color{mauve},
  breaklines=true,
  breakatwhitespace=true,
  tabsize=3
}

\begin{document}

\maketitle

\doublespacing

\section{Statement of the Problem}

The problem to be solved is to present a matrix of images using python. This matrix of images contains images of students who are currently participating in ECE 528 Winter semester. The matrix is to be generated using a simple text file which contains entries student names and a URL where they have posted a photo of themselves. The result is a matrix of these images with the students name underneath the image.

Some problems:
- Students had incorrect URLs
- Students had images of the same file name

\section{Description of Solution}

\singlespacing
\begin{lstlisting}

void setupTestScores(int n, int x[], int min, int max)
float getAve(int n, int x[])
float getStdDev(int n, int x[])
void printArray(int n, int x[], char label[])

\end{lstlisting}
\doublespacing

The function \texttt{setupTestScores} creates N random numbers between 50 and 100 and stores the results in the array \texttt{x[]}. The functions \texttt{getAve} and \texttt{getStdDev} compute and return the average and standard deviation, respectively. In both functions, \texttt{x[]} is the incoming data array and \texttt{n} is the dimension of the array. The main programming construct used here is a FOR loop to compute the required sums. Finally, a \texttt{main()} function was written to create the data and call the functions.

\section{Testing and Output}

To test the program, I tried 2 data sets of test scores with \(n = 10\) and \(n = 20\) students. The results are shown below:

\singlespacing
\begin{lstlisting}
 Test Scores: 91 55 60 81 94 66 53 83 84 85 
           n: 10
     Average: 75.20
Standard Dev: 15.23

 Test Scores: 58 93 59 77 74 84 64 69 54 69 63 68 91 90 53
           n: 15
     Average: 71.07
Standard Dev: 13.40
\end{lstlisting}
\doublespacing

I verified by hand that the results are correct. Note that although I tested on 2 relatively small data sets, the program will work for a data set of any size. In the case of a very large data set, though, I wouldn’t display all of the data to the screen. Rather, I would just display the final results.

\pagebreak

\section{Code}

All code generated within scope of this assignment can be found in the Assignment 2 folder in the GitHub repository:
\url{https://github.com/mikebowyer/ECE528_Assignments}.


\end{document}

