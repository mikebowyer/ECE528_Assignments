\documentclass[12pt, letterpaper, final, onecolumn, titlepage] {article}

\usepackage{enumerate}
\usepackage{graphicx}
\usepackage{listings}
\usepackage{color}
\usepackage{setspace}
\usepackage[margin=1in]{geometry}
\usepackage{mathtools}
\usepackage{amsmath}
\usepackage{hyperref}

\title{ECE 528: Cloud Computing \\
	\vspace{1.5cm}
   		\begin{center}\includegraphics{umlogo} \end{center}
	\vspace{1.5cm}
	\textbf{Assignment \#2} \\
Python Image Download and Display}
	
\author{Michael Bowyer}

\date{\today}

\definecolor{dkgreen}{rgb}{0,0.6,0}
\definecolor{gray}{rgb}{0.5,0.5,0.5}
\definecolor{mauve}{rgb}{0.58,0,0.82}

\lstset{frame=tb,
  language=C,
  aboveskip=3mm,
  belowskip=3mm,
  showstringspaces=false,
  columns=flexible,
  basicstyle={\small\ttfamily},
  numbers=none,
  numberstyle=\tiny\color{gray},
  keywordstyle=\color{blue},
  commentstyle=\color{dkgreen},
  stringstyle=\color{mauve},
  breaklines=true,
  breakatwhitespace=true,
  tabsize=3
}

\begin{document}

\maketitle

\doublespacing

\section{Statement of the Problem}

The problem to be solved is to present a matrix of images using python. This matrix of images contains images of students who are currently participating in ECE 528 Winter semester. The matrix is to be generated using a simple text file which contains entries of student names and a URL where they have posted a photo of themselves. The result is a matrix of these images with the students name underneath the image.

There is the additional requirement for this assignment to extend the implementation created to solve the above problem in some interesting way. Using a face detection algorithm on these student images was the selected functionality. This face detection algorithm simple looks at an image, attempts to find faces inside of the image, and if a face is found then a bounding box of the face is then drawn on top of the image.

\pagebreak
\section{Description of Solution}

The solution for the described problem was completed using a single entrypoint python program which takes three arguments as input prior to it's execution. The usage of the program and its input arguments are described below.

\singlespacing
\begin{lstlisting}
	usage: Assignment2.py [-h] [--student-list STUDENT_LIST] [--output-image-dir OUTPUT_IMAGE_DIR] [-fd]

	Read in student list csv file, then plot all head shots with names. Optionally run facial detection.
	
	optional arguments:
	  -h, --help            show this help message and exit
	  --student-list STUDENT_LIST
							Path to student list
	  --output-image-dir OUTPUT_IMAGE_DIR
							Path to directory where to save output student matrix image
	  -fd, --run-face-detection
							Set when you want to run facial detection for each student image
\end{lstlisting}
\doublespacing

The main function of this program is shown in the below code snippet. Most lines are function calls which are defined in other created files under the lib folder, namely student\_info and img\_utils. The student\_info library contains classes for storing and creating new students and the img\_utils library is responsible for downloading, plotting, and running facial detection on images. Each line of the code snippet contains an important functionality of the program and is accompanied by a descriptive comment. Each step of the program is further described in the itemized list below the code snippet.

\singlespacing
\begin{lstlisting}[language=Python]
	from lib import student_info
	from lib import imgs_utils
	
	#1. Grab argument of location of image, and master information csv
		args = parser.parse_args()
		studentListFilePath = args.student_list
		outputImgDir = args.output_image_dir
		if not os.path.exists(outputImgDir):
			os.makedirs(outputImgDir)

	#2. Read in csv file
		studentInfos = student_info.Students(studentListFilePath)

	#3. Download Images and update students image file path
		validStudents = imgs_utils.download_images_update_students(studentInfos, outputImgDir)

	#4. Sort students based on last name
		validStudents.sort(key=lambda x: x.lastname)

	#5. Display images
		imgs_utils.plot_all_students(validStudents)

	#6. Run facial recognition detection
		if(args.run_face_detection):
			imgs_utils.runFacialDetection(validStudents)
	
\end{lstlisting}
\doublespacing
Here is a further description of each step.
\begin{enumerate}
	\item Arugment parsing reads the arguments passed to the program: source student list, output image directory, flag to run facial detection.
	\item Student file reading reads in all student information from the input source student list and stores them in the variable studentInfos.
	\item All student information from studentInfos is then passed to a created function called download\_images\_update\_students(), which downloads all images from the studentInfo, then appends the local path to where the image was stored locally. All students which an image was sucessfully downloaded for are then returned and saved in validStudents.
	\item The students are sorted in place by last name.
	\item All student images are then combined into a single matrix, then this matrix is saved locally.
	\item Finally, if the user wants to run facial detection, then the program will grab the image for the first student in the list, find and plot bounding boxes around any found faces. This image will be displayed in a popup. This popup is shown until the user of the program presses another key which indicates they want to look at the next image. When the user presses a key, the existing popup is replaced with the output of the facial detection on the next students image. This repeats until all images are exhausted or the program is terminated.
\end{enumerate}

\pagebreak
\section{Encountered Problems}
There were a few problems which were encountered while developing this solution. These are briefly described in the following list along with the solution used:
\begin{enumerate}
	\item Students had incorrect URLs - The solution used was to create a list of valid students, which only contained students who an image could be sucessfully downloaded for.
	\item Student images had same file name - the solution for this problem was to check if there was a image already saved locally with the same name which the download image would be saved to. If there was an existing file name with the same name, then the newly downloaded image would be renamed to include the students name. This can be seen in some of the output section debug statements.
	\item Facial detection did not always detect faces - The facial detection algorithm used is imperfect and is not able to detect faces in all images. The best solution found was to play with the detection parameters to try to make it work well for the dataset of student images.
\end{enumerate}

\pagebreak
\section{Testing and Output}
Below is the output log of the program when the user of the program cycled through three student images. There would be further debug statements with "faces found", but the program was terminated early for the sake of brevity. The output matrix of student images can be seen in figure \ref{studentMatrix} and a few example output images from the facial detection algorithm can be be seen in figure \ref{facialDetect}.

\singlespacing
\begin{lstlisting}
	python.exe Assignment2.py --student-list ./StudentInformation.csv --output-image-dir ./StudentImages --run-face-detection
	INFO: Parsing input arguments
	INFO: Input student information file: ./StudentInformation.csv
	INFO: Saving downloaded images to: ./StudentImages
	INFO: Reading in student information
	INFO: Downloading student images locally
	DEBUG: Image for Ahmed Zaki sucessfully Downloaded: meface.jpg
	DEBUG: Image for Michael Bowyer sucessfully Downloaded: BowyerHeadshot_150x150.PNG
	DEBUG: Image for Hamka Maya sucessfully Downloaded: me.png
	DEBUG: Image for Ponkshe Aishwarya sucessfully Downloaded: AP.png
	DEBUG: Image for Brown Patrick sucessfully Downloaded: SMALL.PORTRAIT2.png
	DEBUG: Image for Delisi Justin sucessfully Downloaded: headshot.png
	DEBUG: Image for Daniel Zajac sucessfully Downloaded: dz_150x150.jpg
	DEBUG: Image for Steiner Tristan sucessfully Downloaded: Steiner_photo_150x150.jpg
	DEBUG: Image for VK Reshma sucessfully Downloaded: reshmavk-150x150.png
	DEBUG: Image for Topolovec Kenny sucessfully Downloaded: KennyTopolovec_150_150.jpg
	DEBUG: Image for Benczarski Joe sucessfully Downloaded: joe_150x150.png
	DEBUG: Image for Paul Graff sucessfully Downloaded: PaulGraffPictureForECE528_2.jpg
	DEBUG: Image for Bhopatrao Prachi sucessfully Downloaded: prachi_img_new.png
	DEBUG: Image for Ujjwal Pratul sucessfully Downloaded: image.jpg
	DEBUG: Image for Bhor Sonal sucessfully Downloaded: SonalBhor.png
	DEBUG: Image for Chinnireddy Balaji sucessfully Downloaded: image.png
	DEBUG: Image for Fozey Alqirsh sucessfully Downloaded: fozey.png
	DEBUG: Image for Bokhari Aidan sucessfully Downloaded: HowDoYouDoFellowKids.png
	DEBUG: Image name for Sivagiri Manjunadh already exists, renaming
	DEBUG: Image for Sivagiri Manjunadh sucessfully Downloaded: Sivagiri_Manjunadh_Image.jpg
	DEBUG: Image for Bhatti Tejbir sucessfully Downloaded: IMG.png
	DEBUG: Image for Patel Kush sucessfully Downloaded: kush.png
	DEBUG: Image for Gilkey Kelsey sucessfully Downloaded: KG428.png
	DEBUG: Image for Blum Bruce sucessfully Downloaded: myImage.png
	DEBUG: Image for Khambam Sanjaykumar sucessfully Downloaded: Sanjay.png
	DEBUG: Image for Hourani Hadi sucessfully Downloaded: Hourani.png
	DEBUG: Image for Virigineni Srija sucessfully Downloaded: myimage.jpg
	DEBUG: Image for Patel Amar sucessfully Downloaded: small.png
	INFO: Sorting students by last name
	INFO: Generating matrix of student images
	INFO: Running facial detection on each student image
	DEBUG: Found 1 faces!
	DEBUG: Found 1 faces!
	DEBUG: Found 1 faces!	
\end{lstlisting}
\doublespacing

\begin{figure}[htbp]
	\centerline{\includegraphics{students.png}}
	\caption{Output student image matrix}
	\label{studentMatrix}
\end{figure}

\begin{figure}[htbp]
	\centerline{\includegraphics[scale=.75]{FacialDetectionImages/FacialDetection.png}}
	\caption{Three example output images from facial detection algorithm}
	\label{facialDetect}
\end{figure}

\pagebreak

\section{Code}

All code generated within scope of this assignment can be found in the Assignment 2 folder in the GitHub repository:
\url{https://github.com/mikebowyer/ECE528_Assignments}.


\end{document}

